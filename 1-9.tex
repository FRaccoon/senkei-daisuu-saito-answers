\documentclass[dvipdfmx,uplatex,11pt]{jsarticle}
%
\usepackage[dvipdfmx]{graphicx}
\usepackage{amsmath,amssymb,amsthm}
\usepackage{enumitem}
\usepackage{wrapfig}
\usepackage{bm}
\usepackage{ascmac}
\setcounter{tocdepth}{2}
\usepackage{geometry}
\usepackage{framed}
\usepackage{latexsym}
%
\geometry{left=10mm,right=10mm,top=5mm,bottom=10mm}
%
\begin{document}
%
%
%
\section{第1章:平面と空間のベクトル}
\subsection{p30}
\noindent
問9:$||x||=||y||=||z||=1$のもとで,$\det (x,y,z)$の最大値,最小値を求めよ。
\\
\textsl{Hint}:行列式の図形的な意味から結果は明らかなように思えるが,ここでは角度を媒介変数として計算で求める.\\
\dotfill


$x$,$y$のなす角を$\theta$,$x \times y$と$z$のなす角を$\phi$とする.\\
$\det(x,y,z)\\
=(x \times y,z)\\
=||x|| ||y|| ||z|| \sin \theta \cos \phi$\\
ここで,$\theta$と$\phi$は独立に動くので,\\
$ -1 \le \sin \theta \cos \phi \le 1$が成立する.\\
また,与条件により,$||x||=||y||=||z||=1$なので,$||x|| ||y|| ||z||=1$\\
ゆえに,$-1 \le \det(x,y,z) \le 1$\\
よって,求める最大値は$1$,最小値は$-1$\\

















\end{document}